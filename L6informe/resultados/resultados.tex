\documentclass{article}

\begin{document}

\section*{Setup}

Para la detección de pulsos del PMT se utilizó el osciloscopio (?). Hablar de Setup \& Hold. Contar cómo usamos un generador de pulsos para verificar el funcionamiento. Incluir link github.com/gastonbarboza/muon ? para que Acha vea todo nuestro código.

\section*{Resultados}

Para cada PMT se realizó una calibración de la cantidad de eventos detectados por minuto en función del threshold establecido en el osciloscopio. Cuando los PMTs son expuestos a luz ambiental aún estando apagados, quedan (??) y envían pulsos causados por fotones térmicos. Al dejarlos encerrados en oscuridad, se observó que aún después de (?) días la cantidad de eventos por minuto registrado por cada PMT continuaba disminuyendo, indicando que el PMT continuaba enfriándose, como puede verse en la figura (?). Esto dificulta saber qué tasa de eventos tiene un PMT a un threshold dado, pero la situación se remedia al medir pulsos en coincidencia, como se muestra más adelante.

Insertar figura susodicha; una compilación de todas las curvas de PMT.

Se optó por usar la calibración en la cual los PMTs estuvieron en oscuridad por (?) días. Usándola, se escribió un script que invertía la curva y calculaba, dada una tasa de eventos deseada, qué threshold necesita cada PMT para lograrla. Con esta información, se hizo un barrido de coincidencias entre placas en función de tasa de eventos. Se consideraba una coincidencia si dos pulsos ocurrían distanciados en el tiempo por 17ns y superaban el threshold pedido: este threshold era calculado por el script de manera tal que ambos PMTs disparaban a una tasa conocida. Se eligió la ventana de 17ns por la consideración de que, con un grosor de (?) cm, un muón viajando a una velocidad cerca a la luz tardaría (?) en atravesarlas. (Algo sobre átomos (??)). El resultado fue la curva azul de la figura (?). Luego se expuso los PMTs a luz para ver si el aumento de eventos afectaría las coincidencias; la curva roja de la figura (?) indica que no. Para levantar esta segunda curva, si bien los PMTs estaban ``en caliente'' y disparando con un gran número de eventos, se usó la calibración de los PMTs con (?) días de oscuridad, en el sentido de que se establecieron thresholds que ya no se correspondían con la tasa de eventos reales que el PMT estaba midiendo, sino con la tasa que medirían tras (?) días. Esto se hizo bajo la suposición de que todos los eventos agregados serían de origen térmico y que entonces no contribuirían a las coincidencias, suposición que se vio corroborada.\footnote{Hay que hablar de la fórmula de coincidencias casuales en algún momento.}

Insertar figura maestra

Hablar sobre falta de meseta o presencia de meseta en la curva.

\end{document}
