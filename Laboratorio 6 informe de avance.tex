\documentclass[12pt,a4paper]{article}

%Packages

%%% Geometria y fuente %%%
\usepackage[utf8]{inputenc}
\usepackage[spanish,es-noquoting]{babel}
\usepackage{geometry}
\geometry{a4paper,left=20mm,right=20mm,top=25mm,bottom=25mm}
\usepackage{bookman} % font
\usepackage{authblk} % estilo de titulo, autor y afiliación
\renewcommand*{\Authfont}{\small\itshape}
\renewcommand*{\Affilfont}{\small\itshape}
\usepackage{setspace} % para modificar el interlineado
%\setlength\parindent{0pt}  elimina sangría de todos los parrafos


%%% Mathematical tools %%%
\usepackage{marvosym}
\usepackage{amsmath}
\usepackage{mathrsfs}
\usepackage{mathtools}
\numberwithin{equation}{section}
\usepackage{amssymb}
\usepackage{bm}


%%% Box and colors %%%
\usepackage{color}
\usepackage{tcolorbox}
\newtcolorbox{boxumen}{colback=white,colframe=teal,boxrule=1pt}
\newtcolorbox{boxquation}{colback=white,colframe=black,boxrule=1pt}


%%% Capital Letter %%%
\usepackage{lettrine} % letra capital}
\setcounter{DefaultLines}{4}
\setlength{\DefaultFindent}{7pt}
\setlength{\DefaultNindent}{0pt}
\renewcommand{\LettrineFontHook}{\usefont{U}{yinit}{m}{n}}
\renewcommand{\DefaultLoversize}{-0.70}
\usepackage{yfonts}
%Primero se declara el paquete lettrine y el numero de renglones que debe abarcar la inicial. En seguida DefaultFindent, la distancia de la inicial a la letra siguiente en el primer renglón y DefaultNindent, la distancia que se desplaza a la derecha del inicio del primer renglón, los renglones subsecuentes que abarca la capitular.Después se declara la font a utilizar, en este caso yinit, con unos parámetros que la describen, y finalmente el tamaño de la letra.


%%% Pagestyle %%%
\usepackage{fancyhdr}

\fancypagestyle{lab}{
\fancyhf{}
\rhead{{\color{brown!60!black}\Large\Coffeecup}}
\lhead{\textit{\small{Laboratorio 6}}}
\fancyfoot{}
%\lfoot{\tiny{}}
\rfoot{\thepage}
}
\fancypagestyle{informe}{
	\fancyhf{}
	\rhead{{\color{brown!60!black}\Large\Coffeecup}}
	\lhead{\textit{\small{Informe de Avance}}}
	\fancyfoot{}
	%\lfoot{\tiny{}}
	\rfoot{\thepage}
}


%Title


\title{\vspace{-20pt}\Large{\textbf{\textcolor{teal}{Construcción de detectores de muones con centellador plástico para la caracterización de detectores de partículas por Cherenkov en agua}}}}

\date{}

\author{Barboza Gastón y Codina Tomás} 
\affil{Instituto de Astronomía y Física del Espacio FCEyN, UBA}
\affil{barboza.gaston@gmail.com, tomycodina@gmail.com}




%Document
\begin{document}


\maketitle
\thispagestyle{lab}

\begin{center}
\vspace{-40pt}\rule{\textwidth}{0.2pt}
\end{center}

\setstretch{1.5} % Interlineado 1.5

\begin{boxumen}
insert abstract here
\end{boxumen}

\section{Introducción}


Los rayos cósmicos (RC) llegan a la tierra desde todas direcciones
del espacio con energías que van desde alrededor de 1 GeV hasta cientos
de EeV (1018eV ). Los detectores de RC son utilizados por muchos
experimentos que necesitan observar estos rayos con distintos propósitos;
por ejemplo el Observatorio LAGO los utiliza para el estudio de la
actividad solar, o el Observatorio Pierre Auger para investigar el origen
de los RC ultraenergéticos.

Este tipo de observatorios pueden utilizar
uno o muchos (arreglo) detectores individuales, cuyas características
deben ser estudiadas para determinar sus condiciones de diseño. Para
este tipo de desarrollo se utilizan detectores auxiliares que permiten
identificar la trayectoria y naturaleza de alguna partícula que entra al
detector, lo cual permite estudiar el tipo de señal que este genera. Un
tipo particular de detector auxiliar es el comúnmente llamado ''paleta
centelladora'', un detector de reducidas dimensiones que consta de
un centellador plástico con un detector óptico adosado que registra los
pulsos de luz producidos en el centellador al ser atravesado por una
partícula cargada.

El elemento básico de estas paletas centelladoras consta de una
barra centelladora de sección rectangular (20 cm x 4 cm) recorrida en
su longitud por una fibra óptica ''waveshifter'' que colecta la luz producida por el centellador, cambiando su longitud de onda y guiándola
hacia un tubo fotomultiplicador (PMT). Dicho esto, el objetivo del trabajo a realizar en laboratorio 6 consiste en construir y caracterizar dos de estos elementos básicos. Para dicha construcción se utilizó la tecnología del detector AMIGA del Observatorio Pierre Auger\cite{pierreauger}. 

\section{Montaje Experimental y Adquisición}

El PMT de los detectores resulta ser muy sensible a la luz, por ello se necesita un entorno completamente oscuro para realizar las mediciones. Debido a esto, previo a crear los detectores, se construyó una caja de madera de llxllxll $ m^3 $ con tapa y totalmente cubierta de aluminio, con dos agujeros para la entrada y salida de cables. Para los primeros, se utilizó un conector ¿? de 3 patas, por dos de ellas ingresaban la alimentación de los detectores (positivo y negativo), mientras que la tercer pata tenía un LED del lado externo de la caja, conectado al circuito interno que indicaba si circulaba o no corriente por los PMT's. Para los cables de salida, se usó un tubo de cañerías de PVC curvo. 
%Luego, los lugares más propensos a dejar ingresar luz, como por ejemplo, los bordes de las tapas, fueron recubiertos con felpa negra.
Finalmente, como mecanismo de seguridad se utilizaron dos switch's ubicados en los extremos de la caja, los cuales cortaban la corriente del circuito interno de los detectores al abrirse la tapa y cerraban el circuito al tapar la caja. La caja puede verse en la imagen xxx\\

IMAGEN\\

 
Luego, para los detectores se utilizaron cuatro placas centelladoras plásticas de 20x4 $ cm^2 $, pintadas con XXX para XXX, las cuales tienen una guía en el medio por donde se introduce la fibra óptica. Para aumentar la señal se utilizaron dos placas para cada detector, ubicandolas una encima de la otra alineadas con respecto a sus guías y adheridas con cinta de papel.

%como muestra la figura YYY.\\
%
%FIGURA\\
 
La señal lumínica se transportó por dos fibras ópticas ''Waveshifter'' en cada detector, para aumentar la señal. Uno de sus bordes ingresaba a uno de los 16 pines que tenían los fotomultiplicadores XXX, los cuales estaban enchufados a unas placas de alto voltaje (H.V.) con un conector de salida de señal disponible para cada pin. Los H.V. de las placas se varíaron durante las distintas pruebas con un potenciometro para encontrar la respuesta óptima en ambos detectores.
%los voltajes fueron de xxx a xxx para la placa correspondiente a uno de los PMT's y xxx a xxx para la del segundo. 

Luego, las placas se conectaban a una fuente continua de $ 12V $ por medio de la ficha de 3 puntas y finalmente el conector de la placa correspondiente al pin en uso se soldaba a un cable BNC, que salía por el tubo de PVC, e ingresaba en un osciloscopio XXX el cual se manejó con una computadora mediante scripts en Python. 

Dado que se realizaron dos tipos de mediciones, uno correspondiente a la caracterización de los detectores por separado y otro la medición de coincidencias, Las posiciones de las placas variaron en uno u otro caso. Para las curvas de calibración (donde se midió el Rate de muones en función del voltaje de salida), la caja se ubicó con su lado mayor en forma horizontal y ambos detectores en distintas posiciones pero unidos por los cables de la fuente de $ 12V $. Por otro lado, para conincidencias se utilizaron dos configuraciones distintas. En la primera, se dejo la caja en la misma posición y se solaparon ambos centelladores uno sobre el otro. En la segunda, para variar la distancia entre centelladores, se utilizó la caja en forma vertical introduciendo un banco de madera fabricado en la tornería del IAFE para apoyar encima de él uno de los detectores, con esto se logró tener ambas placas distanciadas XXX cm. Todo el circuito de la configuración horizontal y vertical puede apreciarse en la imagen XXX\\

IMAGEN (a(horizontal) y b(vertical))\\

Para la adquisición de datos, el osciloscopio se utilizó de dos formas diferentes dependiendo del tipo de medición. Para la calibración, los pulsos de voltajes salientes del detector en cuestión se llevaban a los canales 1 y 2 por medio de un BNC. Luego, se discriminaban los pulsos según un \textbf{Threshold mínimo}, los pulsos que pasaban dicho umbral, eran registrados por el osciloscopio. La forma de contar cuantos pulsos se detectaban era mediante la cantidad de veces que el dispositivo triggereaba la señal. Con estas herramientas, se variaba de forma automática el thresshold mínimo cada un cierto tiempo, recolectando el número total de triggers durante ese periodo. Este último número se graficaba en función del threshold mínimo con lo que se obtenía la curva de calibración. 

Por otro lado, para detectar coincidencias, se utilizó la función \textbf{Setup and Hold} del osciloscopio el cual utilizaba ambos canales (uno para cada detector) tomando el canal 1 de ellos como referencia. Esta función tomaba los valores de referencia que superasen un cierto threshold para iniciar un temporizador, luego, se fijaba en el canal 2 si aparecía un pulso que superase otro threshold y este dentro de una dada ventana temporal. De esta forma, si esta condiciones se cumplían el oscoloscopio trigereaba, lo que considerabamos un evento. De esta forma, teniendo en cuenta los retardos entre cables y velocidades de respuesta de los PMT's, se  fijaba una ventana temporal razonable y luego se barrían distintos Thresholds de forma automática para obtener finalmente el gráfico de eventos en función de Thresholds o, utilizando la curva de calibración, Rates.

\section{Resultados y discusiones}

BLA BLA BLA

\section{Conclusiones}

BLA BLA BLA

\section{Agradecimientos}

\begin{thebibliography}{9}
	
	\bibitem{pierreauger}
	Claude Itzykson \& Jean-Bernard Zuber,
	\textit{Quantum Field Theory},
	Capítulos 3.2 y 3.3.
	
	\bibitem{Itzqui}
	Lewis H.Ryder,
	\textit{Quantum Field Theory},
	Capítulo 9.
	
	\bibitem{Itzqui}
	Michael E. Peskin \& Daniel V. Schoeder,
	\textit{An introduction to Quantum Field Theory},
	Capítulos 6, 7 y 10.
	
\end{thebibliography}

\end{document}